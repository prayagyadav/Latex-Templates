% Winter Theme
% Author: Prayag Yadav
% Website: https://www.prayagyadav.com

% Content Tex

\documentclass[11pt,xcolor={dvipsnames},hyperref={luatex,pdfpagemode=UseNone,hidelinks,pdfdisplaydoctitle=true},usepdftitle=false]{beamer}
%\usepackage{presentation}
%\usepackage{jupyter}
\usepackage{presentation}
\usepackage[breakable]{tcolorbox}
\usepackage{minted}
\usepackage{fancyvrb,xcolor}
\usepackage{tikz-feynman}
\usepackage{graphicx}
\usepackage{smartdiagram}
\usepackage{xcolor-material}

\usepackage{amssymb}
\usepackage{pifont}

% To use with pgf plots
%\usepackage{pgfplots}
%\pgfplotsset{compat=1.14}
%textwidth is 4.64658 inches


% Enter title of presentation PDF:
\hypersetup{pdftitle={Title}}

%Code blocks

\newcommand{\boxempty}{\color{black}$\square$}
\newcommand{\boxtick}{\color{black}\rlap{\raisebox{0.3ex}{\hspace{0.4ex}\tiny \ding{52}}}$\square$}
\newcommand{\boxcross}{\color{black}\rlap{\raisebox{0.3ex}{\hspace{0.4ex}\scriptsize \ding{56}}}$\square$}

\newenvironment{VerbExample}
{\example  \semiverbatim}
{\endsemiverbatim \example}

\newminted{bash}{gobble=2}
\newminted{python}{gobble=2}
%\newminted{python}{gobble=2,linenos}
\newminted{text}{gobble=2}

\begin{document}

\section{First Section }

\begin{frame}
	\heading{First Section}
\end{frame}

\begin{frame}[fragile] % don't forget fragile when using code boxes or feynman diagrams
	\frametitle{Insert Heading}

	\begin{itemize}
		\item You can enter your \textbf{text} here.
		\item And, of course, all the $ f_O\mathbb{R}mu=lae$.
	\end{itemize}

		\begin{graybox}[Code block name]
		\begin{bashcode}
		prayag@hp-fedora:~$
		\end{bashcode}
	\end{graybox}

\end{frame}

\begin{frame}
	\frametitle{Feynman Diagrams}
	\begin{tcolorbox}[title=Can also add feynman diagrams]
		\begin{figure}
			\centering
			$$e^+e^- \rightarrow ZH \rightarrow \mu^+ \mu^- + X$$
			\feynmandiagram [horizontal=a to b] {
				i1 [particle=\(e^{-}\)] -- [fermion] a -- [fermion] i2 [particle=\(e^{+}\)],
				a -- [photon, edge label=\(Z\)] b,
				f1 -- [photon,edge label=\( Z \)] b -- [scalar] f2 [particle=\(H (Recoil)\)],
				f1--[fermion]m1[particle=\( \mu^{-} \)],
				m2[particle=\( \mu^{+} \)] --[fermion]f1,
			};
		\end{figure}
	\end{tcolorbox}
\end{frame}

\end{document}
